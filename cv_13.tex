%%%%%%%%%%%%%%%%%%%%%%%%%%%%%%%%%%%%%%%%%
% Wilson Resume/CV
% XeLaTeX Template
% Version 1.0 (22/1/2015)
%
% This template has been downloaded from:
% http://www.LaTeXTemplates.com
%
% Original author:
% Howard Wilson (https://github.com/watsonbox/cv_template_2004) with
% extensive modifications by Vel (vel@latextemplates.com)
%
% License:
% CC BY-NC-SA 3.0 (http://creativecommons.org/licenses/by-nc-sa/3.0/)
%
%%%%%%%%%%%%%%%%%%%%%%%%%%%%%%%%%%%%%%%%%

%----------------------------------------------------------------------------------------
%	PACKAGES AND OTHER DOCUMENT CONFIGURATIONS
%----------------------------------------------------------------------------------------

\documentclass[10pt]{article} % Default font size
\usepackage[super]{nth}
\usepackage{paracol}

\input{structure.tex} % Include the file specifying document layout
\newcommand{\sqbulle}{~\vrule height 1ex width .8ex depth -.2ex} % Custom square bullet point
%----------------------------------------------------------------------------------------

\newcommand{\titledate}[2][2.5in]{%
  \noindent%
  \begin{tabular}{@{}p{#1}@{}}
    \\ \hline \\[-.75\normalbaselineskip]
    Date
  \end{tabular}
  \begin{flushright}
  \begin{tabular}{@{}p{#1}@{}}
    \\ \hline \\[-.75\normalbaselineskip]
    #2
  \end{tabular} \hspace{1in}
  \end{flushright}
  }

\begin{document}

%----------------------------------------------------------------------------------------
%	NAME AND CONTACT INFORMATION
%----------------------------------------------------------------------------------------

\title{Giovanni Guerrieri} % Print the main header

%------------------------------------------------

\parbox{0.5\textwidth}{ % First block
\begin{tabbing} % Enables tabbing
\hspace{3cm} \= \hspace{4cm} \= \kill % Spacing within the block
{\bf Address} \> Via Guarino Guarini 11, \\ % Address line 1
\> 41124, Modena, Italy \\ % Address line 2
{\bf Date of Birth} \> February 22$^{nd}$ 1995 \\ % Date of birth 
{\bf Nationality} \> Italian % Nationality
\end{tabbing}}
\hfill % Horizontal space between the two blocks
\parbox{0.5\textwidth}{ % Second block
\begin{tabbing} % Enables tabbing
\hspace{3cm} \= \hspace{4cm} \= \kill % Spacing within the block
%{\bf Home Phone} \> +0 (000) 111 1111 \\ % Home phone
{\bf Mobile Phone} \> +39 349 412 9947 \\ % Mobile phone
{\bf Email} \> \href{mailto:giovanni.guerrieri00@gmail.com}{giovanni.guerrieri00@gmail.com} \\ % Email address
{\bf Personal website} \> \href{https://www.giovanniguerrieri.it/}{giovanniguerrieri.it}
\end{tabbing}}

%----------------------------------------------------------------------------------------
%	PERSONAL PROFILE
%----------------------------------------------------------------------------------------

%\section{Personal Profile}

%Lorem ipsum dolor sit amet, consectetur adipiscing elit. Duis elementum nec dolor sed sagittis. Cras justo lorem, volutpat mattis lacus vel, consequat aliquam quam. Interdum et malesuada fames ac ante ipsum primis in faucibus. Integer blandit, massa at tincidunt ornare, dolor magna interdum felis, ac blandit urna neque in turpis.

%----------------------------------------------------------------------------------------
%	EDUCATION SECTION
%----------------------------------------------------------------------------------------

\section{Education}

\tabbedblock{
\bf{2022-present} \> CERN Doctoral Student\\
\>\textbf{CERN, Genève, Switzerland}\\[5pt]

\>\begin{minipage}[c][][b]{0,7\textwidth}
1-year contract, INFN funded. \\ \\ 
\end{minipage}
\\[5pt]
}

\tabbedblock{
\bf{2020-present} \> PhD in Physics \\
\>\textbf{\href{https://df.units.it/en}{University of Trieste} - INFN Scholarship}\\[5pt]
%\>First Class - 80\% Average\\
%\>\+
\>\begin{minipage}[c][][b]{0,7\textwidth}
 ``Search for new physics in $t\bar{t}$ resonance final states''. \\ \emph{Supervisor: Dott. Michele Pinamonti, Co-supervisor:~Dott.~Giancarlo~Panizzo} \\ \\
\vspace{-6mm}

Main topics of the courses attended: electroweak physics, flavor physics, semiconductor detectors, machine learning, advanced Bayesian methods. \\ \\ 
\end{minipage}
}

\tabbedblock{
\bf{2017-2019} \> Master Degree in Nuclear and Subnuclear Physics - Grade: 110/110 e Lode \\
\>\textbf{\href{https://df.units.it/en}{University of Trieste} }\\[5pt]
%\>First Class - 80\% Average\\
%\>\+
%\textit{Third Year Project - 89\% awarded `Project of the Year 2007'} 
\>\begin{minipage}[c][][b]{0,7\textwidth}
 ``Charged Triple Gauge Couplings at present and future colliders''. \\ \emph{Supervisor: Prof. Marina Cobal, Co-supervisor: Dott. Giancarlo Panizzo}\\
\vspace{-2mm}
 
Study of W+W− production at present and future hadron and lepton colliders, Large Hadron Collider (LHC), the High Luminosity LHC (HL-LHC) and the leptonic Future Circular Collider (FCC-ee), based on Monte Carlo simulation to extract expected sensitivity to Triple Gauge Couplings (TGCs).\\ \\
\end{minipage}

}

%------------------------------------------------

\tabbedblock{
\bf{2014-2017} \> Bachelor Degree in Physics \\ \>\textbf{\href{http://www.fim.unimore.it/site/en/home.html}{University of Modena and Reggio Emilia}}\\[5pt]
%\>First Class - 80\% Average\\
%\>\+
%\textit{Third Year Project - 89\% awarded `Project of the Year 2007'} 
\>\begin{minipage}[c][][b]{0,7\textwidth}
 ``Misure di campi magnetici al TEM mediante fasci elettronici strutturati''. \\ \emph{Supervisor: Prof. Stefano Frabboni, Co-supervisor: Dott. Gian Carlo Gazzadi}\\ \\
\end{minipage}

}

\tabbedblock{
\bf{2009-2014} \> High School Diploma (Scientific Studies)\\
\>\textbf{Liceo Scientifico Alessandro Tassoni, Modena}
\\[5pt]
}

%----------------------------------------------------------------------------------------
%	EMPLOYMENT HISTORY SECTION
%----------------------------------------------------------------------------------------

%\section{Employment History}

%\job
%{Sep 2011 -}{Present}
%{Lehman Brothers, 1234 Mario Park, San Francisco, CA, United States}
%{http://www.lehmanbrothers.com}
%{Senior Developer / Technical Team Lead}
%{Lorem ipsum dolor sit amet, consectetur adipiscing elit. Duis elementum nec dolor sed sagittis. Cras justo lorem, volutpat mattis lacus vel, consequat aliquam quam. Interdum et malesuada fames ac ante ipsum primis in faucibus.\\
%\rule{0mm}{5mm}\textbf{Technologies:} Ruby on Rails 2.3, Amazon EC2, NoSQL data stores, memcached, collaborative matching, Facebook Graph API.}

%------------------------------------------------

\vspace{15mm}


\section{Professional Experience}

\job
{July \nth{25} 2022 -}{July \nth{29} 2022}
{\textbf{CODATA-RDA Research Data Science Advanced Workshops}}
{https://indico.ictp.it/event/9935/}
{At ICTP, Trieste, Italy}
{\begin{minipage}[c][][b]{0,7\textwidth}
Organization of activities; python basics, columnar analysis.
\end{minipage}
%\rule{0mm}{5mm}\textbf{Technologies:} Ruby on Rails 2.3, Amazon EC2, NoSQL data stores, memcached, collaborative matching, Facebook Graph API.
}

%\job
%{March \nth{1} 2022 -}{Present}
%{\textbf{CERN Doctoral Student}}
%{https://indico.cern.ch/event/1025052/}
%{At CERN, Genève, Switzerland}
%{\begin{minipage}[c][][b]{0,7\textwidth}
%1-year contract, INFN funded
%\end{minipage}
%\rule{0mm}{5mm}\textbf{Technologies:} Ruby on Rails 2.3, Amazon EC2, NoSQL data stores, memcached, collaborative matching, Facebook Graph API.
%}

\job
{July 2021 -}{Present}
{\textbf{INFN Cloud contributor}}
{https://www.cloud.infn.it/}
{At INFN}
{\begin{minipage}[c][][b]{0,7\textwidth}
 Development of Identity and Access Management (IAM) methods for ATLAS Open Data; integration of ATLAS Open Data analysis tools within the INFN Cloud infrastructure; testing of new cloud architectures.
 \end{minipage}
%\rule{0mm}{5mm}\textbf{Technologies:} Ruby on Rails 2.3, Amazon EC2, NoSQL data stores, memcached, collaborative matching, Facebook Graph API.
}


\job
{August 2020 -}{November 2020}
{\textbf{INFN Scolarship}}
{https://jobs.dsi.infn.it/borseassegni/pdf/getfile.php?filename=21706_graduatoria_2649.pdf}
{At CERN, Genève, Switzerland}
{\begin{minipage}[c][][b]{0,7\textwidth}
 Exploration of a multi-cloud (SaaS + IaaS) approach to be adapted to modest scenarios where analysis pipelines can be deployed with the use of Virtual Machines and Containers, providing analysis environments and protocols. Maintenance of such tool within the ATLAS Open Data project. 
\end{minipage}
%\rule{0mm}{5mm}\textbf{Technologies:} Ruby on Rails 2.3, Amazon EC2, NoSQL data stores, memcached, collaborative matching, Facebook Graph API.
}


\job
{May 2019 -}{Present}
{\textbf{Member of the ATLAS Collaboration}}
{https://atlas.cern/}
{At Università degli Studi di Udine, Udine, Italy}
{
\begin{itemize}
    \item ATLAS author (since January \nth{1}, 2022)
    \item Working with the ATLAS Top \& Exotics groups
    \item Working with the ATLAS Outreach \& Open Data group
\end{itemize}
%\rule{0mm}{5mm}\textbf{Technologies:} Ruby on Rails 2.3, Amazon EC2, NoSQL data stores, memcached, collaborative matching, Facebook Graph API.
}

\section{Other Experience}


\job
{March 2022 -}{June 2022}
{\textbf{Innovation For Change (I4C) program}}
{https://www.innovation4change.eu/2022-snam/}
{At CERN Idea Square, Collège des Ingénieurs, and Politecnico di Torino}
{\begin{minipage}[c][][b]{0,7\textwidth}
Development of sustainable solutions for future energy transition challenges.
\end{minipage}
%\rule{0mm}{5mm}\textbf{Technologies:} Ruby on Rails 2.3, Amazon EC2, NoSQL data stores, memcached, collaborative matching, Facebook Graph API.
}

\job
{2020 -}{Present}
{\textbf{Web developer}}
{}
{Independent}
{}
\vspace{-10mm}
\job
{2013 -}{Present}
{\textbf{Personal tutor for high school students}}
{}
{Independent}


\section{Outreach}

\job
{Sept. \nth{24} 2022}{}
{\textbf{Il Bosone di Higgs e le frontiere dell'universo}}
{https://www.youtube.com/watch?v=RTKjdCMpTSE}
{In Trieste, Italy}
{\begin{minipage}[c][][b]{0,7\textwidth}
Outreach seminar for the \nth{10} annyversary of the Higgs boson discovery, part of the cycle "Dialoghi sul futuro" during the Trieste Next event
\end{minipage}
%\rule{0mm}{5mm}\textbf{Technologies:} Ruby on Rails 2.3, Amazon EC2, NoSQL data stores, memcached, collaborative matching, Facebook Graph API.
}


\job
{March 2022 -}{Present}
{\textbf{Official CERN Guide}}
{https://visit.cern/}
{At CERN, Genève, Switzerland}
{\begin{minipage}[c][][b]{0,7\textwidth}
 Organization of visits for the general public, universities and schools to several CERN interest points such as ATLAS, CMS, the CERN Control Center, the CERN Data Center, etc.
\end{minipage}
%\rule{0mm}{5mm}\textbf{Technologies:} Ruby on Rails 2.3, Amazon EC2, NoSQL data stores, memcached, collaborative matching, Facebook Graph API.
}

\job
{2021 -}{2022}
{\textbf{Trieste Next}}
{https://www.triestenext.it/}
{In Trieste, Italy}
{\begin{minipage}[c][][b]{0,7\textwidth}
 Organization of INFN activities and explanation to general public
\end{minipage}
%\rule{0mm}{5mm}\textbf{Technologies:} Ruby on Rails 2.3, Amazon EC2, NoSQL data stores, memcached, collaborative matching, Facebook Graph API.
}

\job
{2020 -}{2022}
{\textbf{International Masterclass on particle physics}}
{https://atlasud.uniud.it/online-masterclasses-2022}
{In Udine, Italy}
{\begin{minipage}[c][][b]{0,7\textwidth}
 Organization of the virtual infrastructure used for the lectures. \\
 Tutoring of the students during the hands-on sessions.
\end{minipage}
%\rule{0mm}{5mm}\textbf{Technologies:} Ruby on Rails 2.3, Amazon EC2, NoSQL data stores, memcached, collaborative matching, Facebook Graph API.
}






\section{Publications}
\href{http://cdsweb.cern.ch/search?ln=en&cc=ATLAS+Papers&p=year\%3A2022&action_search=Search&op1=a&m1=a&p1=&f1=&c=ATLAS+Papers&c=&sf=&so=d&rm=&rg=100&sc=0&of=hb}{50 papers} papers as part of the ATLAS Collaboration.\\

\textbf{``Regression Deep Neural Networks for top-quark-pair resonance searches in the dilepton channel''}\\
Authors: \textbf{G. Guerrieri} \\
\href{http://dx.doi.org/10.1393/ncc/i2022-22110-0}{\emph{Il Nuovo Cimento} 45 C (2022) 110} \\

\textbf{``A proposal for Open Access data and tools multi-user deployment using ATLAS Open Data for Education''}\\
Authors: A. Sanchez Pineda and \textbf{G. Guerrieri} \\
\href{http://dx.doi.org/10.1051/epjconf/202125101008}{\emph{EPJ Web Conf.} 251 01008 (2021)} \\


\section{Talks and Posters}
\textbf{\href{https://indico.in2p3.fr/event/27968/contributions/115537}{FCC-France \& Italy Workshop, Lyon, France}} \\
``Forward/Backward asymmetries at the FCC-ee'' \\
21-23 November 2022  \\

\textbf{\href{https://conference.ippp.dur.ac.uk/event/925/contributions/5840/}{TOP 2022 Conference, Durham, UK}} \\
``First Run 3 data/MC plots for the measurement of the top-quark pair production cross-section in pp collisions at centre-of-mass energy of 13.6 TeV with the ATLAS experiment at the LHC'' \\
4-9 September 2022  \\

\textbf{\href{https://indico.cern.ch/event/1064327/contributions/4884215/}{FCC Week 2022, Paris, France}} \\
``Measuring $Z$ boson couplings to bottom quarks at FCCee'' \\
 May 30 - June 3, 2022  \\

\textbf{107$^{\circ}$ Congresso Nazionale Società Italiana di Fisica} \\
``Regression Deep Neural Networks for top-quark-pair resonance searches'' \\
13-17 September 2021  \\

\textbf{106$^{\circ}$ Congresso Nazionale Società Italiana di Fisica} \\ 
``Charged Triple Gauge Couplings at present and future colliders'' \\
14-18 September 2020  \\



\section{Description of Scientific activity}
\paragraph{PhD:} My PhD research activity is being carried on in the context of experimental particle physics, within the ATLAS experiment at the LHC collider, and with the FCC Collaboration, at CERN. \\
ATLAS (A Toroidal LHC ApparatuS) is a multi-purpose detector; its aim is to take advantage of the unprecedented energy and luminosity available at the LHC and observe phenomena that involve the production of highly massive particles. \\
FCC-ee (Future electron-positron Circular Collider) on the other side, is a novel highest-luminosity energy frontier collider, meant to address the open questions of modern physics. It will be a general precision instrument for the continued in-depth exploration of nature at the smallest scales, optimised to study with high precision the $Z$, $W$, Higgs and top particles, with samples of $5 \times 10^{12} Z$ bosons, $10^8 W$ pairs, $10^6$ Higgs bosons and $10^6$ top quark pairs. \\
I am currently undergoing 1 year as a Doctoral Student at CERN, contributing both to the ATLAS experiment and to the FCC-ee program.
My thesis project focuses on the measurement of the top-quark pair (``$t\bar{t}$  resonance'') cross section during the early Run 3 of LHC, with $pp$ collisions at centre-of-mass energy of 13.6 TeV. The measurement is performed in both dilepton and single lepton channels, and constitutes one of the first analyses of the ATLAS collaboration during Run 3, contributing to both the physics programme and several detector performance studies. Moreover, I am pursuing the search for new massive particles decaying to $t\bar{t}$  resonances exploiting the usage of Deep Neural Networks. \\
Thanks to these novel tools the analysis is expected to outperform those which use more traditional methods, and to provide crucial improvements in sensitivity to physics beyond the Standard Model (BSM) at the LHC. \\
On the other hand, I am continuing the activity of main maintainer of the ATLAS Open Data computational tools, which allow not-expert users to deploy complex analysis infrastructures, using as a backbone the cloud providers available on the market. My effort consists in updating the resources made available to the public and providing support with the installation of new infrastructures.\\
Finally, I am contributing to the measurement of $Z$ boson couplings to bottom quarks at FCC-ee; this analysis has the main purpose of probing such interactions through a competitive determination of $A_{0,b}^{FB}$, improving the precision of the measurement, but most importantly assessing the future detector requirements and recommendations. \\ \\





\begin{paracol}{2}
\sqbulle \hspace{2mm} \textbf{Programming Languages}\\
\begin{itemize}
\vspace{-6mm}
\item \textbf{C++}\\%- 2015-Present \\
\vspace{-6mm}
\item \textbf{Python}\\%- 2015-2016\\
\vspace{-6mm}
\item \textbf{Bash}\\%- 2018-Present\\
\vspace{-6mm}
\item \textbf{R}\\%- 2018-Present\\
\end{itemize}
\switchcolumn
\sqbulle \hspace{2mm} \textbf{Languages}
\begin{itemize}
%\vspace{-6mm}
\item \textbf{Italian}- Native \\
\vspace{-6mm}
\item\textbf{English}- Proficient \\
\vspace{-6mm}
\item\textbf{French}- Intermediate \\
\vspace{-6mm}
\item\textbf{Spanish}- Intermediate
\end{itemize}
\end{paracol}



%------------------------------------------------
%------------------------------------------------

%----------------------------------------------------------------------------------------
%	INTERESTS SECTION
%----------------------------------------------------------------------------------------

\section{Personal Interests}

\interestsgroup{
\interest{Ex pro-volleyball player}
\interest{Video editing}
\interest{Web Development}
}




%\vspace{25mm}
%\titledate{Giovanni Guerrieri}


\end{document}